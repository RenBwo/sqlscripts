% Created 2017-12-06 三 16:38
\documentclass[11pt]{article}
\usepackage[utf8]{inputenc}
\usepackage[T1]{fontenc}
\usepackage{fixltx2e}
\usepackage{graphicx}
\usepackage{longtable}
\usepackage{float}
\usepackage{wrapfig}
\usepackage{rotating}
\usepackage[normalem]{ulem}
\usepackage{amsmath}
\usepackage{textcomp}
\usepackage{marvosym}
\usepackage{wasysym}
\usepackage{amssymb}
\usepackage{hyperref}
\tolerance=1000
\author{sirius}
\date{\today}
\title{设计分析}
\hypersetup{
  pdfkeywords={},
  pdfsubject={},
  pdfcreator={Emacs 25.3.2 (Org mode 8.2.10)}}
\begin{document}

\maketitle
\tableofcontents

\section{}
\label{sec-1}
\section{主要问题}
\label{sec-2}
\subsection{1.工资类型}
\label{sec-2-1}
工资根据人员可以分为行政人员 计件人员 参照行政管理的人员 车间管理者
根据计算方法,可以分为计时工资$\backslash$计件工资$\backslash$系数工资
\subsection{2.税前工资构成}
\label{sec-2-2}
\subsubsection{2.1 行政人员工资=计时基本工资+奖项-扣减项}
\label{sec-2-2-1}
计时基本工资: 根据出勤计时*工资计算标准
奖项:   需要计入税前工资的奖励部分
扣减项: 需要计入税前工资的扣减部分
\subsubsection{2.2 计件人员工资=计件基本工资+计时基本工资+奖项-扣减项}
\label{sec-2-2-2}
计件基本工资:ERP计件工资数据引入,手工EXCE整理的计件工资数据
计时基本工资:计件人员有一部分按计时计算的工资
奖项:   需要计入工资的增加项
扣减项: 需要计入工资的扣减项

\subsubsection{2.3 车间管理者工资=本部门平均工资*系数+奖项-扣减项}
\label{sec-2-2-3}
系数:根据绩效考核确定的系数
\subsection{3.计算方法}
\label{sec-2-3}
\subsubsection{非系数人员工资计算公式}
\label{sec-2-3-1}
\subsubsection{系数工资人员计算公式}
\label{sec-2-3-2}
\subsection{4.操作说明}
\label{sec-2-4}
\subsubsection{部门 与人员 信息维护,包括部门与人员的变动,工资类型$\backslash$所属部门$\backslash$职务$\backslash$职等$\backslash$特殊工龄$\backslash$}
\label{sec-2-4-1}
\subsubsection{工资计算常量设置,包括}
\label{sec-2-4-2}
% Emacs 25.3.2 (Org mode 8.2.10)
\end{document}
